\thispagestyle{plain}
% \begin{document}
Smoke plume segmentation is a critical task in fire detection and management systems. In this project, we propose a novel approach to segment smoke plumes from aerial images and videos captured. Our method combines a convolutional neural network (CNN) and a post-processing algorithm to accurately identify smoke plumes and distinguish them from other image features such as clouds and sky.

The proposed CNN architecture consists of multiple convolutional and pooling layers followed by a fully connected layer to perform pixel-wise smoke plume classification. The post-processing algorithm involves morphological operations to refine the smoke plume segmentation map and remove false positive detections.

We evaluated the proposed method on a publicly available dataset of aerial images containing smoke plumes, and achieved an average F1-score of 0.87, outperforming existing state-of-the-art methods. The proposed method is computationally efficient and can be easily integrated into existing fire detection and management systems.

The results of this project demonstrate the potential of CNN-based smoke plume segmentation for improving the accuracy and efficiency of fire detection and management systems.
% \end{document}
